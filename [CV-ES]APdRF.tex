\documentclass[changecolor={240, 95, 64}]{cv}
\usepackage{multicol}

\begin{document}

\pagestyle{empty}

\header{}{Alberto Pérez de Rada Fiol}{
  \faEnvelope \FAspace apdrf.94@gmail.com \sep
  \faMobile \FAspace +34 646 613 189}{
  \faLinkedinSquare \FAspace /in/albertopdrf \sep
  \faGithub \FAspace /AlbertoPdRF
}

\textit{Físico nuclear con 5 años de experiencia investigadora. Interesado en campos como la energía nuclear, la innovación tecnológica y la programación. Actualmente estoy finalizando mis estudios de Doctorado en Física, a falta sólo de la defensa de la Tesis Doctoral.}

\section*{Experiencia profesional}
\begin{tabularcv}
  2018-2022   &   \worktitle{Investigador contratado como Titulado Superior (53 meses)}{Centro de Investigaciones Energéticas, Medioambientales y Tecnológicas (CIEMAT)}
  \newline Realización de mi Tesis Doctoral
  \\
  2020   &   \worktitle{Práctica (3 meses)}{Beca MLH impulsada por GitHub}
  \\
  2017-2018   &   \worktitle{Encargado de la zona de Madrid (4 meses)}{Yobbiks}
  \\
  2017   &   \worktitle{Práctica (6 meses)}{CIEMAT}
  \newline Realización de mi Trabajo de Fin de Máster
  \\
  2016   &   \worktitle{Práctica (5 meses)}{Universidad Autónoma de Madrid}
  \newline Proyecto de Innovación Docente C\textunderscore 08.15: ``SPOC para la enseñanza de ``Laboratorio de Química Teórica Aplicada""
  \\
  2015-2016   &   \worktitle{Promotor de la marca ``V-Cube" en El Corte Inglés Preciados (1 mes)}{Compudid}
\end{tabularcv}

\section*{Educación}
\begin{tabularcv}
  2017-presente   &   \worktitle{Programa de Doctorado en Física}{Universidad Complutense de Madrid}
  \newline Título de la Tesis Doctoral: ``Espectroscopía de neutrones $\beta$-retardados de $^{85,86}$As con MONSTER"
  \\
  2016-2017   &   \worktitle{Máster Universitario en Física Nuclear (8,35)}{Universidad Autónoma de Madrid}
  \newline Título del Trabajo de Fin de Máster: ``Caracterización y simulaciones de un detector CLYC para espectroscopía de rayos gamma y neutrones"
  \\
  2012-2016   &   \worktitle{Grado en Física (6,73)}{Universidad Autónoma de Madrid}
\end{tabularcv}

\section*{Cursos}
\begin{tabularcv}
  2022   &   \worktitle{ARIEL-H2020 International On-Line School on Nuclear Data: The Path from the Detector to the Reactor Calculation - NuDataPath (32 horas)}{CIEMAT}
  \\
  2019   &   \worktitle{Protección Radiológica (12 horas)}{CIEMAT}
  \\
  2018   &   \worktitle{Protección de Riesgos Laborales (12 horas)}{CIEMAT}
  \\
  2014   &   \worktitle{Curso de inmersión en Lengua Inglesa (40 horas)}{Universidad Internacional Menéndez Pelayo}
\end{tabularcv}

\section*{Idiomas}
\begin{itemize}
  \item Castellano: nativo
  \item Inglés: nivel alto
  \item Catalán: nativo
\end{itemize}

\section*{Artículos}
\begin{tabularcv}
  2023   &   E. Mendoza \textit{et al.}, <<Neutron capture measurements with high efficiency detectors and the Pulse Height Weighting Technique>>, \emph{Nuclear Instruments and Methods in Physics Research A} \link{https://doi.org/10.1016/j.nima.2022.167894}
  \\
  2023   &   J. Plaza \textit{et al.}, <<Thermal neutron background at Laboratorio Subterráneo de Canfranc (LSC)>>, \emph{Astroparticle Physics} \link{https://doi.org/10.1016/j.astropartphys.2022.102793}
  \\
  2022   &   A. K. Mistry \textit{et al.}, <<The DESPEC setup for GSI and FAIR>>, \emph{Nuclear Instruments and Methods in Physics Research A} \link{https://doi.org/10.1016/j.nima.2022.166662}
  \\
  2018   &   T. Martínez \textit{et al.}, <<Characterization of a CLYC detector for underground experiments>>, \emph{Nuclear Instruments and Methods in Physics Research A} \link{https://doi.org/10.1016/j.nima.2018.07.087}
\end{tabularcv}

\section*{Congresos}
\begin{tabularcv}
  2022   &   <<$\beta$-delayed neutron spectroscopy of $^{85,86}$As with MONSTER>>, \emph{European Nuclear Physics Conference 2022}
  \\
  2022   &   <<$\beta$-delayed neutron spectroscopy of $^{85}$As with MONSTER>>, \emph{15$^{th}$ International Conference on Nuclear Data for Science and Technology}
  \\
  2018   &   <<Caracterización y simulaciones de un detector CLYC para espectroscopía de rayos gamma y neutrones>>, \emph{44ª Reunión Anual Sociedad Nuclear Española}
  \\
  2017   &   <<Caracterización y simulaciones de un detector CLYC para espectroscopía de rayos gamma y neutrones>>, \emph{IX CPAN DAYS}
\end{tabularcv}

\section*{Habilidades de software}
\begin{itemize}
  \item Lenguajes de programación / tecnologías: C++, ROOT, Geant4, Python, LaTeX, Ruby, Rails, JavaScript, React, Node.js, Express, HTML, CSS, MySQL, PostgreSQL, MongoDB, MatLab
  \item Sistemas operativos: Linux, Windows
  \item Herramientas de productividad: G Suite, Office 365
\end{itemize}

\section*{Reconocimientos}
\begin{tabularcv}
  2018   &   Finalista Premio Sociedad Nuclear Española para Trabajos y Proyectos Fin de Máster 2018
  \\
  2014-2016   &   Tres veces campeón de España resolviendo el cubo de Rubik con los pies
  \\
  2013, 2015   &   Dos veces campeón de España resolviendo el ``Pyraminx", un puzle secuencial tipo Rubik
\end{tabularcv}

\section*{Voluntariados}
\begin{tabularcv}
  2019   &   \worktitle{Gestor de Proyectos del Equipo de ``Software" (WST, por sus siglas en inglés)}{Asociación Mundial del Cubo (WCA, por sus siglas en inglés)} \link{https://www.worldcubeassociation.org}
  \\
  2018-2019   &   \worktitle{Fundador y Presidente}{Asociación Madrileña de Speedcubing (AMS)} \link{https://www.speedcubingmadrid.org}
  \\
  2018   &   \worktitle{Organizador del WCA European Championship 2018}{WCA} \link{https://www.worldcubeassociation.org/competitions/Euro2018}
  \\
  2017-2019   &   \worktitle{Miembro de la Junta Directiva}{WCA}
  \newline Nombrado Secretario de la Asociación en 2018
  \\
  2017-2019   &   \worktitle{Miembro del Comité de Reglamento (WRC, por sus siglas en inglés)}{WCA}
  \\
  2016-2019   &   \worktitle{Delegado}{WCA}
  \\
  2016-2018   &   \worktitle{Miembro de la Junta Directiva}{Asociación Española del Cubo de Rubik (AECR)} \link{http://asociacionrubik.es}
\end{tabularcv}

\section*{Trayectoria laboral}

Mi trayectoria como Investigador contratado como Titulado Superior se detalla a continuación.

\subsection*{Participación y/o gestión en proyectos de I+D+i}

Durante mi trayectoria como Investigador contratado como Titulado Superior he participado en los siguientes proyectos de I+D+i.

\begin{itemize}
  \item Proyectos europeos:
        \begin{itemize}
          \item Proyecto ENSAR2, del Programa Marco Horizonte 2020: desde julio de 2018 hasta
                agosto de 2021.
          \item Proyecto EURAD, del Programa Marco Horizonte 2020: desde junio de 2019 hasta
                noviembre de 2022.
          \item Proyecto SANDA, del Programa Marco Horizonte 2020: desde septiembre de 2019
                hasta noviembre de 2022.
        \end{itemize}
  \item Proyectos nacionales:
        \begin{itemize}
          \item Proyecto del Plan Nacional de Física de Partículas de I+D+i FPA2016-76765-P: desde julio de 2018 hasta diciembre de 2018.
          \item Proyecto del Programa Estatal de Generación del Conocimiento y Fortalecimiento Científico y Tecnológico del Sistema de I+D+i PGC2018-096717-B-C21: desde enero de 2019 hasta noviembre de 2022.
          \item Proyecto de I+D+i para la realización de «Pruebas de Concepto»PDC2021-120828-I00: desde enero de 2022 hasta noviembre de 2022.
          \item Proyecto de Generación de Conocimiento PID2021-123100NB-I00, en el marco del Programa Estatal para Impulsar la Investigación Científico-Técnica y su Transferencia: desde octubre de 2022 hasta noviembre de 2022.
        \end{itemize}
  \item Contratos con la industria:
        \begin{itemize}
          \item Con ENRESA, en el proyecto “Transmutación de radionucleidos de vida larga como soporte a la gestión de residuos radioactivos de alta actividad” (2016-2020). Desde julio de 2018 hasta julio de 2020.
        \end{itemize}
\end{itemize}

\subsection*{Experiencia en el desarrollo de actividades técnicas y de investigación}

Poseo una dilatada experiencia en experimentos de física nuclear en el marco de la investigación en física nuclear o tecnologías nucleares, así como un gran conocimiento del campo de la instrumentación nuclear. Mis experiencia en estas actividades se detalla a continuación:

\begin{itemize}
  \item Investigación en física nuclear o tecnologías nucleares. Cabe destacar los siguientes experimentos:
        \begin{itemize}
          \item La preparación, la realización y el análisis completo del experimento ``Caracterización del detector MONSTER con los neutrones $\beta$-retardados del decaimiento del $^{85}$As" en el Laboratorio del Acelerador de la Universidad de Jyväskylä (Finlandia) para obtener el espectro de emisión de neutrones $\beta$-retardados de $^{85,86}$As.
          \item La participación en la medida para la caracterización del fondo intrínseco de un detector CLYC para su uso en experimentos subterráneos en el Laboratorio Subterráneo de Canfranc.
        \end{itemize}
  \item Instrumentación nuclear.
        \begin{itemize}
          \item Caracterización de detectores. Tengo una gran experiencia en la caracterización de detectores de neutrones, rayos gamma y electrones. He caracterizado la respuesta de varios tipos de detectores, entre los que cabe destacar:
                \begin{itemize}
                  \item Los centelleantes orgánicos líquidos que forman el detector MOdular Neutron time-of-flight SpectromeTER (MONSTER) para la detección de neutrones y rayos gamma y plásticos para la detección de electrones.
                  \item Los cristales centelleantes inorgánicos CLYC y LaBr$_3$.
                  \item Detectores de semiconductores de germanio de alta pureza tipo clover.
                \end{itemize}
          \item Simulación de detectores. También tengo una gran experiencia en la simulación de detectores de neutrones, rayos gamma y electrones con Geant4. En particular, he simulado la respuesta a neutrones, rayos gamma y electrones con varios tipos de detectores:
                \begin{itemize}
                  \item Centelleantes orgánicos (plásticos y líquidos). Destaca la simulación de MONSTER, compuesto actualmente por más de 60 celdas de centelleantes líquidos, incluyendo la simulación de tres modelos de celdas, dos de ellos desarrollados entre el CIEMAT y la empresa española Scientifica International.
                  \item Centelleantes inorgánicos, destacando la simulación de los cristales tipo CLYC y LaBr$_3$.
                  \item Detectores de semiconductores de germanio de alta pureza tipo clover.
                \end{itemize}
        \end{itemize}
\end{itemize}

Para la obtención de los resultados de los experimentos de física nuclear en que he participado, así como para la caracterización de la instrumentación nuclear usada, he desarrollado rutinas de ajustes de pulsos, construcción de eventos y varios tipos de análisis.

\subsection*{Experiencia en actividades relacionadas con el funcionamiento y puesta en marcha de laboratorios e instalaciones científico-técnicas y de investigación}

Cuento con experiencia en el funcionamiento de cuatro instalaciones científicas: el Laboratorio de Datos Nucleares y el Laboratorio de Patrones Neutrónicos del CIEMAT, el Laboratorio Subterráneo de Canfranc y el Laboratorio del Acelerador de la Universidad de Jyväskylä (Finlandia). En estas instalaciones, he realizado tareas que incluyen la calibración de detectores, el montaje de dispositivos experimentales, la monitorización de experimentos, la gestión de la toma de datos con digitalizadores y el procesado de datos con rutinas de ajustes de pulsos y construcción de eventos para su posterior análisis.

\subsection*{Experiencia en actividades de divulgación y comunicación}

Los resultados de los distintos proyectos en los que he participado se han presentado en varios congresos a nivel nacional e internacional, de acuerdo a las listas de publicaciones y comunicaciones a congresos que se han detallado en las correspondientes secciones de este curriculum vitae.

\end{document}
