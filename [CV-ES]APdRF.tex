\documentclass[changecolor={240, 95, 64}]{cv}
\usepackage{multicol}

\begin{document}

\pagestyle{empty}

\header{}{Alberto Pérez de Rada Fiol}{
  \faEnvelope \FAspace apdrf.94@gmail.com \sep
  \faMobile \FAspace +34 646 613 189}{
  \faLinkedinSquare \FAspace /in/albertopdrf \sep
  \faGithub \FAspace /AlbertoPdRF
}

\section*{Experiencia profesional}
\begin{tabularcv}
  % 2020   &   \worktitle{Práctica}{Beca MLH impulsada por GitHub}
  % \\
  2018-2022   &   \worktitle{Investigador predoctoral}{Centro de Investigaciones Energéticas, Medioambientales y Tecnológicas (CIEMAT)}
  \newline Realización de mi Tesis Doctoral
  \\
  2017-2018   &   \worktitle{Encargado de la zona de Madrid}{Yobbiks}
  \\
  2017   &   \worktitle{Práctica}{CIEMAT}
  \newline Realización de mi Trabajo de Fin de Máster
  \\
  2016   &   \worktitle{Práctica}{Universidad Autónoma de Madrid}
  \newline Proyecto de Innovación Docente C\textunderscore 08.15: ``SPOC para la enseñanza de ``Laboratorio de Química Teórica Aplicada""
  \\
  2015-2016   &   \worktitle{Promotor de la marca ``V-Cube" en El Corte Inglés Preciados}{Compudid}
\end{tabularcv}

\section*{Educación}
\begin{tabularcv}
  2017-presente   &   \worktitle{Programa de Doctorado en Física}{Universidad Complutense de Madrid}
  \newline Título de la Tesis Doctoral: ``Espectroscopía de neutrones $\beta$-retardados de $^{85,86}$As con MONSTER"
  \\
  2016-2017   &   \worktitle{Máster Universitario en Física Nuclear}{Universidad Autónoma de Madrid}
  \newline Título del Trabajo de Fin de Máster: ``Caracterización y simulaciones de un detector CLYC para espectroscopía de rayos gamma y neutrones"
  \\
  2012-2016   &   \worktitle{Grado en Física}{Universidad Autónoma de Madrid}
\end{tabularcv}

\section*{Cursos}
\begin{tabularcv}
  2022   &   \worktitle{ARIEL-H2020 International On-Line School on Nuclear Data: The Path from the Detector to the Reactor Calculation - NuDataPath}{CIEMAT}
  \\
  2014   &   \worktitle{Curso de inmersión en Lengua Inglesa}{Universidad Internacional Menéndez Pelayo}
\end{tabularcv}

\section*{Idiomas}
\begin{itemize}
  \item Castellano: nativo
  \item Inglés: nivel alto
  \item Catalán: nativo
\end{itemize}

\section*{Artículos}
\begin{tabularcv}
  2023   &   E. Mendoza \textit{et al.}, <<Neutron capture measurements with high efficiency detectors and the Pulse Height Weighting Technique>>, \emph{Nuclear Instruments and Methods in Physics Research A} \link{https://doi.org/10.1016/j.nima.2022.167894}
  \\
  2023   &   J. Plaza \textit{et al.}, <<Thermal neutron background at Laboratorio Subterráneo de Canfranc (LSC)>>, \emph{Astroparticle Physics} \link{https://doi.org/10.1016/j.astropartphys.2022.102793}
  \\
  2022   &   A. K. Mistry \textit{et al.}, <<The DESPEC setup for GSI and FAIR>>, \emph{Nuclear Instruments and Methods in Physics Research A} \link{https://doi.org/10.1016/j.nima.2022.166662}
  \\
  2018   &   T. Martínez \textit{et al.}, <<Characterization of a CLYC detector for underground experiments>>, \emph{Nuclear Instruments and Methods in Physics Research A} \link{https://doi.org/10.1016/j.nima.2018.07.087}
\end{tabularcv}

\section*{Congresos}
\begin{tabularcv}
  2022   &   <<$\beta$-delayed neutron spectroscopy of $^{85,86}$As with MONSTER>>, \emph{European Nuclear Physics Conference 2022}
  \\
  2022   &   <<$\beta$-delayed neutron spectroscopy of $^{85}$As with MONSTER>>, \emph{15$^{th}$ International Conference on Nuclear Data for Science and Technology}
  \\
  2018   &   <<Caracterización y simulaciones de un detector CLYC para espectroscopía de rayos gamma y neutrones>>, \emph{44ª Reunión Anual Sociedad Nuclear Española}
  \\
  2017   &   <<Caracterización y simulaciones de un detector CLYC para espectroscopía de rayos gamma y neutrones>>, \emph{IX CPAN DAYS}
\end{tabularcv}

\section*{Habilidades de software}
\begin{itemize}
  \item Lenguajes de programación / tecnologías: C++, ROOT, Geant4, Python, LaTeX, Ruby, Rails, JavaScript, React, Node.js, Express, HTML, CSS, MySQL, PostgreSQL, MongoDB, MatLab
  \item Sistemas operativos: Linux, Windows
  \item Herramientas de productividad: G Suite, Office 365
\end{itemize}

\section*{Reconocimientos}
\begin{tabularcv}
  2018   &   Finalista Premio Sociedad Nuclear Española para Trabajos y Proyectos Fin de Máster 2018
  \\
  2014-2016   &   Tres veces campeón de España resolviendo el cubo de Rubik con los pies
  \\
  2013, 2015   &   Dos veces campeón de España resolviendo el ``Pyraminx", un puzle secuencial tipo Rubik
\end{tabularcv}

\section*{Voluntariados}
\begin{tabularcv}
  2019   &   \worktitle{Gestor de Proyectos del Equipo de ``Software" (WST, por sus siglas en inglés)}{Asociación Mundial del Cubo (WCA, por sus siglas en inglés)} \link{https://www.worldcubeassociation.org}
  \\
  2018-2019   &   \worktitle{Fundador y Presidente}{Asociación Madrileña de Speedcubing (AMS)} \link{https://www.speedcubingmadrid.org}
  \\
  2018   &   \worktitle{Organizador del WCA European Championship 2018}{WCA} \link{https://www.worldcubeassociation.org/competitions/Euro2018}
  \\
  2017-2019   &   \worktitle{Miembro de la Junta Directiva}{WCA}
  \newline Nombrado Secretario de la Asociación en 2018
  \\
  2017-2019   &   \worktitle{Miembro del Comité de Reglamento (WRC, por sus siglas en inglés)}{WCA}
  \\
  2016-2019   &   \worktitle{Delegado}{WCA}
  \\
  2016-2018   &   \worktitle{Miembro de la Junta Directiva}{Asociación Española del Cubo de Rubik (AECR)} \link{http://asociacionrubik.es}
\end{tabularcv}

\end{document}
